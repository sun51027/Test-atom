% !TeX root = ../main.tex

\chapter{Injection mechanism of tail wave injection}
在上面章節中我們發現,雖然都是透過衝擊波產生的密度斜坡進行LWFA的注入,顯然這項結果與過去多數的研究結果並不相似。
我也成功還原出比較接近於多數研究的模擬,並且從各項電子品質進行分析與比較。
分析出此種注入的機制以及其優點與缺點。
% In the above section, we found that although the LWFA injection was carried out through the density slope generated by the shock wave, it is obvious that this result is not similar to that of most previous studies.I have also successfully reproduced simulations that are close to most studies, and analyzed and compared various electronic qualities.The mechanism of this injection and its advantages and disadvantages are analyzed.
\section{尾波注入和加載注入的電子特性比較}%{Comparison of electron injection characteristics between tail wave injection and beamloading injection}



\section{Differences in tilted shock-front performance}


\section{Discussion}
