% !TeX root = ../main.tex

\chapter{緒論}





\section{雷射電漿加速器}
自從 Tajima 和 Dawson 於 1979 年首次理論預測基於等離子體的電子加速以來,已經發表了大量的實驗架設和改進的模型。當前的模型預測在實際實驗條件下能產生約 10GeV 電子束,勞倫斯伯克利國家實驗室 (LBNL) 的一個團隊在 4.2 .GeV突破當前實驗實現的峰值電子能量記錄僅使用 9 厘米。
因此,這些機器目前正在達到與經典射頻 (r.f.) 加速器相當的電子能量,並且顯著減少了佔用空間。作為比較,斯坦福線性加速器中心 (SLAC) 的線性相干光源 (LCLS) 使用 1 公里長的加速器來實現 10 GeV 峰值能量的電子束。
然而RF加速器在其他電子束品質方面勝過雷射電漿的電子源,這些參數決定了加速電子束在自由電子激光器 (FEL) 或碰撞實驗中的可用性。其中之一是電子束參數的穩定性。
1994 年在英國摩德納等人的盧瑟福阿普爾頓實驗室。在加州大學洛杉磯分校 (UCLA) 的雷射拍頻狀態中顯示加速尾場存在近 10 年後,在自調變雷射電漿尾場狀態中顯示了峰值能量為 44MeV 的電子。不同之處在於,電子第一次不是從射頻源外部注入,而是在通常被稱為破波注入的過程中從背景電漿中捕捉。
外部注入的必要性也隨之減少,並使基於雷射電漿加速實驗變得更加容易。在過去的二十年中,雷射電漿加速實驗幾乎完全顯示了從背景電漿中捕獲的加速電子。
為了從實驗上研究雷射尾場加速(LWFA)機制,這種方法出現了一些挑戰。如果想研究 LWFA 裝置中注入電子束的演化,則初始分佈至少需要保持不變,如果不知道的話。在通常為幾毫米的小規模加速器實驗中,在改變加速部分的參數的同時創建保持恆定的背景電子的局部注入是非常具有挑戰性的。
無論注入條件如何,與相比於尾場加速中獲得的動量,最初電子在雷射方向上的動量非常小。因此,可用的自由度、加速度長度和等離子體密度,適用於研究縱向動量增益,並且已經開發出模型,與實驗結果有很好的一致性。
\section{自由電子雷射對電子源的限制}
